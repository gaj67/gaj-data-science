\documentclass[a4paper]{article}
\usepackage[a4paper,margin=25mm]{geometry}
\usepackage{graphicx,subcaption}
\usepackage{amsmath,amsfonts}
\usepackage{amssymb}
\usepackage{accents}

\newcommand{\rvec}[1]{\accentset{\leftarrow}{#1}}
\renewcommand{\v}[1]{\mathbf{#1}}
\renewcommand{\c}[1]{{\cal #1}}

\title{Gaussian Restricted Boltzmann Classifier}
\author{G.A. Jarrad}

\begin{document}
\maketitle
\numberwithin{equation}{section}
\numberwithin{figure}{section}
\numberwithin{table}{section}
\section{Definition}\label{sec:intro}
Consider a restricted Boltzmann machine (RBM) with a real-valued input layer,
a binary-valued hidden layer, and a binary-valued output layer, as shown in 
Figure~\ref{fig:rbm}.
A suitable energy function is given by
\begin{eqnarray}
    E(\v{x},\v{h},\v{y}; \Theta) & = & \frac{1}{2}\|\v{x}-\v{a}\|^2
    - \v{b}^T\v{h} - \v{h}^T W\v{x} - \v{c}^T\v{y} - \v{h}^T U\v{y}\,,
\label{eq:energy}
\end{eqnarray}
with input feature vector $\v{x}=(x_1,x_2,\ldots,x_F)\in\mathbb{X}\subseteq\mathbb{R}^F$, 
hidden binary vector $\v{h}=(h_1,h_2,\ldots,h_H)\in\mathbb{H}=\{0,1\}^H$, and output 
binary vector $\v{y}=(y_1,y_2,\ldots,y_C)\in\mathbb{Y}=\{0,1\}^C$.
The model parameters are $\Theta=(\v{a}, \v{b}, \v{c}, W, U)$.
The joint probability of $\v{x}$, $\v{y}$ and $\v{h}$ is then
\begin{eqnarray}
    p(\v{x},\v{h},\v{y}\;|\;\Theta) & = & \frac{e^{-E(\v{x},\v{h},\v{y}; \Theta)}}
{\int_{\mathbb{X}}\sum_{\v{h}'\in\mathbb{H}}\sum_{\v{y}'\in\mathbb{Y}}
e^{-E(\v{x}',\v{h}',\v{y}'; \Theta)}\,d|\v{x}'|
}
\,,
\end{eqnarray}
which is intractible to compute in general.

In order to turn the RBM into a restricted Boltzmann classifier (RBC), let us now suppose that the binary vector $\v{y}$
is really a one-in-$C$ vector of $C-1$ zeros and a single one, restricted to the set
$\mathbb{Y}'=\{\v{y}\in\mathbb{Y}\;|\;\sum_{k=1}^{C}y_k=1\}$. Then there is a one-to-one correspondence between each
vector $\v{y}\in\mathbb{Y}'$ and some scalar $y\in\{1,2,\ldots,C\}$, such that, for example,
the term $U\v{y}$ selects the $y$-th column of $U$, denoted by $\v{u}_y$.
Hence we obtain a final mapping to a multinomial output, suitable for a classifier.
The joint probability then becomes
\begin{eqnarray}
     p(\v{x},\v{h},y\;|\;\Theta) &=&
\frac{
    e^{-\frac{1}{2}\|\v{x}-\v{a}\|^2+\v{b}^T\v{h}+\v{h}^T W\v{x}+c_y+\v{h}^T \v{u}_y}
}
{\int_{\mathbb{X}}\sum_{y'=1}^{C}\sum_{\v{h}'\in\mathbb{H}}
   e^{-\frac{1}{2}\|\v{x}'-\v{a}\|^2+\v{b}^T\v{h}'+\v{h}'^T W\v{x}'+c_{y'}+\v{h}^T \v{u}_{y'}}
   \,d|\v{x}'|
}
\nonumber\\&=&
\frac{
    e^{c_y-\frac{1}{2}\|\v{x}-\v{a}\|^2}
\prod_{i=1}^{H}e^{h_i(b_i+ \v{w}_i^T\v{x}+u_{iy})}
}
{\int_{\mathbb{X}}\sum_{y'=1}^{C}
   e^{c_{y'}-\frac{1}{2}\|\v{x}'-\v{a}\|^2}
  \prod_{i=1}^{H}\left[1+e^{b_i+ \v{w}_i^T\v{x}+u_{iy}}\right]
  \,d|\v{x}'|
}
\,,
\end{eqnarray}
where $\v{w}_i^T$ is the $i$-th row of $W$.
The discriminative form of the RBC can then be specified as
\begin{eqnarray}
    p(y\;|\;\v{x},\Theta) & = & 
\frac{\sum_{\v{h}'\in\mathbb{H}}p(\v{x},\v{h}',y\;|\;\Theta)}
{\sum_{y'=1}^{C}\sum_{\v{h}'\in\mathbb{H}}p(\v{x},\v{h}',y'\;|\;\Theta)}
\nonumber\\&=&
\frac{
    e^{c_y-\frac{1}{2}\|\v{x}-\v{a}\|^2}\prod_{i=1}^{H}\left[1+e^{b_i+ \v{w}_i^T\v{x}+u_{iy}}\right]
}
{\sum_{y'=1}^{C}
   e^{c_{y'}-\frac{1}{2}\|\v{x}-\v{a}\|^2}\prod_{i=1}^{H}\left[1+e^{b_i+ \v{w}_i^T\v{x}+u_{iy'}}\right]
}
\nonumber\\&=&
\frac{
   e^{c_y}\prod_{i=1}^{H}\left[1+e^{b_i+ \v{w}_i^T\v{x}+u_{iy}}\right]
}
{
\sum_{y'=1}^{C} 
e^{c_{y'}}\prod_{i=1}^{H}\left[1+e^{b_i+ \v{w}_i^T\v{x}+u_{iy'}}\right]
}
\,,
\label{eq:p_y_x}
\end{eqnarray}
which is a nonlinear form of logistic classifier.

Now, the bipartite restriction depicted in Figure~\ref{fig:rbm} ensures that $\v{x}$ and $y$
are conditionally independent given $\v{h}$.
Observe, for instance, that
\begin{eqnarray}
   p(\v{x}\;|\;\v{h},y,\Theta) & = &
\frac{ p(\v{x},\v{h},y\;|\;\Theta)}
{ \int_{\mathbb{X}}p(\v{x}',\v{h},y\;|\;\Theta)\,d|\v{x}'|}
\nonumber\\&=&
\frac{
    e^{-\frac{1}{2}\|\v{x}-\v{a}\|^2+\v{b}^T\v{h}+\v{h}^T W\v{x}+c_y+\v{h}^T \v{u}_y}
}
{
\int_{\mathbb{X}}
    e^{-\frac{1}{2}\|\v{x}'-\v{a}\|^2+\v{b}^T\v{h}+\v{h}^T W\v{x}'+c_y+\v{h}^T \v{u}_y}
\,d|\v{x}'|
}
\nonumber\\&=&
\frac{
    e^{-\frac{1}{2}\|\v{x}-\v{a}\|^2+\v{h}^T W\v{x}}
}
{
\int_{\mathbb{X}}
    e^{-\frac{1}{2}\|\v{x}'-\v{a}\|^2+\v{h}^T W\v{x}'}
\,d|\v{x}'|
}
\nonumber\\&=&
\frac{
    e^{-\frac{1}{2}\|\v{x}-\v{a}-W^T\v{h}\|^2+\v{h}^T W\v{a}+\frac{1}{2}\v{h}^T WW^T \v{h}}
}
{
    \int_{\mathbb{X}}
    e^{-\frac{1}{2}\|\v{x}-\v{a}-W^T\v{h}\|^2+\v{h}^T W\v{a}+\frac{1}{2}\v{h}^T WW^T \v{h}}
    \,d|\v{x}'|
}
\nonumber\\&=&
\frac{
    e^{-\frac{1}{2}\|\v{x}-\v{a}-W^T\v{h}\|^2}
}
{
    \int_{\mathbb{X}}e^{-\frac{1}{2}\|\v{x}-\v{a}-W^T\v{h}\|^2}\,d|\v{x}'|
}
\nonumber\\&=& N(\v{x}\;|\;\v{a}+W^T\v{h},I)\,.
\end{eqnarray}
Hence, $\v{x}$ is conditionally normally distributed with mean $\v{a}+W^T\v{h}$ and unit spherical 
variance $I$ (the identity matrix).

Similarly, observe that
\begin{eqnarray}
    p(y\;|\;\v{x},\v{h},\Theta) & = & 
\frac{p(\v{x},\v{h},y\;|\;\Theta)}
{\sum_{y'=1}^{C}p(\v{x},\v{h},y'\;|\;\Theta)}
\nonumber\\&=&
\frac{
    e^{-\frac{1}{2}\|\v{x}-\v{a}\|^2+\v{b}^T\v{h}+\v{h}^T W\v{x}+c_y+\v{h}^T\v{u}_y}
}
{
   \sum_{y'=1}^{C}
    e^{-\frac{1}{2}\|\v{x}-\v{a}\|^2+\v{b}^T\v{h}+\v{h}^T W\v{x}+c_{y'}+\v{h}^T\v{u}_{y'}}
}
\nonumber\\&=&
\frac{e^{c_y+\v{h}^T\v{u}_y}}
  {\sum_{y'=1}^{C}e^{c_{y'}+\v{h}^T \v{u}_{y'}}}\,.
\end{eqnarray}
This result is just the {\em soft-max} function, or standard logistic classifier.

Conversely, $\v{h}$ depends upon both $\v{x}$ and $\v{y}$ via
\begin{eqnarray}
    p(\v{h}\;|\;\v{x},y,\Theta) & = & 
    \frac{p(\v{x},\v{h},y\;|\;\Theta)}
{\sum_{\v{h}'\in\mathbb{H}}p(\v{x},\v{h}',y\;|\;\Theta)}
\nonumber\\&=&
\frac{
    e^{c_y-\frac{1}{2}\|\v{x}-\v{a}\|^2}
\prod_{i=1}^{H}e^{h_i(b_i+ \v{w}_i^T\v{x}+u_{iy})}
}
{
   e^{c_{y}-\frac{1}{2}\|\v{x}-\v{a}\|^2}
  \prod_{i=1}^{H}\left[1+e^{b_i+ \v{w}_i^T\v{x}+u_{iy}}\right]
}
\nonumber\\&=&
\frac{
\prod_{i=1}^{H}e^{h_i(b_i+ \v{w}_i^T\v{x}+u_{iy})}
}
{
  \prod_{i=1}^{H}\left[1+e^{b_i+ \v{w}_i^T\v{x}+u_{iy}}\right]
}
\nonumber\\&=&
  \prod_{i=1}^{H}p(h_i\;|\;\v{x},y,\Theta)
\,,
\end{eqnarray}
where 
\begin{eqnarray}
  p(h_i=1\;|\;\v{x},y,\Theta) & = & 
  \frac{e^{b_i+\v{w}_i^T\v{x}+u_{iy}}}{1+e^{b_i+\v{w}_i^T\v{x}+u_{iy}}}
~=~\sigma(b_i+\v{w}_i^T\v{x}+u_{iy})
  \,.
\label{eq:ph1}
\end{eqnarray}
This is just the logistic sigmoid function.

\section{Supervised Discriminative Optimisation}
Consider the problem of estimating the RBC parameters $\Theta$ from a data-set of fully labelled feature vectors,
$\c{X}=(\v{x}_1,\v{x}_2,\ldots,\v{x}_N)$, with corresponding labels $\c{Y}=(y_1,y_2,\ldots,y_N)$.
Assuming that the data items are independent, the discriminative likelihood is given by
\begin{eqnarray}
  p(\c{Y}\;|\;\c{X},\Theta) & = & \prod_{d=1}^N p(y_d\;|\;\v{x}_d,\Theta)
\,,
\end{eqnarray}
 and hence, from equation~\eqref{eq:p_y_x}, the average discriminative log-likelihood is given by
\begin{eqnarray}
  \c{L}_{\c{Y}|\c{X}}(\Theta) & = & \frac{1}{N}\ln p(\c{Y}\;|\;\c{X},\Theta)
~=~ 
\frac{1}{N}\sum_{d=1}^N \ln p(y_d\;|\;\v{x}_d,\Theta)
\nonumber\\& = & 
\frac{1}{N}\sum_{d=1}^N\left\{
  \sum_{y'=1}^C \delta_{y',y_d}\ln\left(
   e^{c_{y'}}\prod_{i=1}^{H}\left[1+e^{b_i+ \v{w}_i^T\v{x}_d+u_{iy'}}\right]\right)\right.
\nonumber\\&&
{}-\left.\ln\sum_{y'=1}^{C} 
e^{c_{y'}}\prod_{i=1}^{H}\left[1+e^{b_i+ \v{w}_i^T\v{x}_d+u_{iy'}}\right]
\right\}
\nonumber\\& = & 
\frac{1}{N}\sum_{d=1}^N\left\{
  \sum_{y'=1}^C \delta_{y',y_d}\left(
   c_{y'}+\sum_{i=1}^{H}\ln\left[1+e^{b_i+ \v{w}_i^T\v{x}_d+u_{iy'}}\right]\right)\right.
\nonumber\\&&
{}-\left.\ln\sum_{y'=1}^{C} 
e^{c_{y'}}\prod_{i=1}^{H}\left[1+e^{b_i+ \v{w}_i^T\v{x}_d+u_{iy'}}\right]
\right\}\,.
\end{eqnarray}
Hence, the gradient with respect to $c_y$ is
\begin{eqnarray}
\frac{\partial\c{L}_{\c{Y}|\c{X}}}{\partial c_y}
& = & 
\frac{1}{N}\sum_{d=1}^N\left\{
\delta_{y,y_d}-
\frac{e^{c_{y}}\prod_{i=1}^{H}\left[1+e^{b_i+ \v{w}_i^T\v{x}_d+u_{iy}}\right]}
{\sum_{y'=1}^{C} 
e^{c_{y'}}\prod_{i=1}^{H}\left[1+e^{b_i+ \v{w}_i^T\v{x}_d+u_{iy'}}\right]}
\right\}
\nonumber\\&=&
\frac{1}{N}\sum_{d=1}^N\left\{
\delta_{y,y_d}-p(y\:|\;\v{x}_d,\Theta)
\right\}
\nonumber\\&=&
\frac{N_y}{N}-\frac{1}{N}\sum_{d=1}^N p(y\:|\;\v{x}_d,\Theta)
\,,
\end{eqnarray}
where $N_y$ is the number of data labelled with class $y$.

In order to develop the remaining derivatives, we first observe that
\begin{eqnarray}
\frac{\partial}{\partial\theta_i}\left(
c_{y}+\sum_{i'=1}^{H}\ln\left[1+e^{b_{i'}+ \v{w}_{i'}^T\v{x}_d+u_{i'y}}\right]
\right)
&=&
\frac{\partial}{\partial\theta_i}\ln\left[1+e^{b_{i}+ \v{w}_{i}^T\v{x}_d+u_{iy}}\right]
\nonumber\\&=&
\frac{e^{b_{i}+ \v{w}_{i}^T\v{x}_d+u_{iy}}}
{1+e^{b_{i}+ \v{w}_{i}^T\v{x}_d+u_{iy}}}
\frac{\partial}{\partial\theta_i}(b_{i}+ \v{w}_{i}^T\v{x}_d+u_{iy})
\nonumber\\&=&
p(h_i=1\;|\;\v{x}_d,y,\Theta)
\frac{\partial}{\partial\theta_i}(b_{i}+ \v{w}_{i}^T\v{x}_d+u_{iy})\,,
\end{eqnarray}
from equation~\eqref{eq:ph1},
and then use the fact that 
$\nabla f(\theta)=f(\theta)\nabla\ln f(\theta)$ to deduce that
\begin{eqnarray}
&\frac{\partial}{\partial\theta_i}e^{c_{y}}\prod_{i'=1}^{H}\left[1+e^{b_{i'}+ \v{w}_{i'}^T\v{x}_d+u_{i'y}}\right]&
\nonumber\\&=&\hspace*{-25mm}
e^{c_{y}}\prod_{i'=1}^{H}\left[1+e^{b_{i'}+ \v{w}_{i'}^T\v{x}_d+u_{i'y}}\right]
\frac{\partial}{\partial\theta_i}\left(
c_{y}+\sum_{i'=1}^{H}\ln\left[1+e^{b_{i'}+ \v{w}_{i'}^T\v{x}_d+u_{i'y}}\right]
\right)
\nonumber\\&=&\hspace*{-25mm}
e^{c_{y}}\prod_{i'=1}^{H}\left[1+e^{b_{i'}+ \v{w}_{i'}^T\v{x}_d+u_{i'y}}\right]
p(h_i=1\;|\;\v{x}_d,y,\Theta)
\frac{\partial}{\partial\theta_i}(b_{i}+ \v{w}_{i}^T\v{x}_d+u_{iy})\,.
\end{eqnarray}
Hence, the gradient of the log-likelihood with respect to $u_{iy}$ is
\begin{eqnarray}
\frac{\partial\c{L}_{\c{Y}|\c{X}}}{\partial u_{iy}}
& = & 
\frac{1}{N}\sum_{d=1}^N\left\{\rule{0pt}{5mm}
\delta_{y,y_d}\,p(h_i=1\;|\;\v{x}_d,y,\Theta)\right.
\nonumber\\&&
{}-\left.\frac{e^{c_{y}}\prod_{i'=1}^{H}\left[1+e^{b_{i'}+ \v{w}_{i'}^T\v{x}_d+u_{i'y}}\right]}
{\sum_{y'=1}^{C}e^{c_{y'}}\prod_{i'=1}^{H}\left[1+e^{b_{i'}+ \v{w}_{i'}^T\v{x}_d+u_{i'y'}}\right]}
p(h_i=1\;|\;\v{x}_d,y,\Theta)
\right\}
\nonumber\\&=&
\frac{1}{N}\sum_{d=1}^N\left\{
\delta_{y,y_d}-p(y\;|\;\v{x}_d,\Theta)\right\}\,p(h_i=1\;|\;\v{x}_d,y,\Theta)
\,.
\end{eqnarray}
Similarly, the gradient of the log-likelihood with respect to $b_{i}$ is
\begin{eqnarray}
\frac{\partial\c{L}_{\c{Y}|\c{X}}}{\partial b_{i}}
& = & 
\frac{1}{N}\sum_{d=1}^N\left\{\sum_{y'=1}^{C}
\delta_{y',y_d}\,p(h_i=1\;|\;\v{x}_d,y',\Theta)\right.
\nonumber\\&&
{}-\left.\frac{\sum_{y'=1}^{C}e^{c_{y'}}\prod_{i'=1}^{H}\left[1+e^{b_{i'}+ \v{w}_{i'}^T\v{x}_d+u_{i'y'  }}\right]
\,p(h_i=1\;|\;\v{x}_d,y',\Theta)}
{\sum_{y'=1}^{C}e^{c_{y'}}\prod_{i'=1}^{H}\left[1+e^{b_{i'}+ \v{w}_{i'}^T\v{x}_d+u_{i'y'}}\right]}
\right\}
\nonumber\\&=&
\frac{1}{N}\sum_{d=1}^N\sum_{y'=1}^{C}\left\{
\delta_{y',y_d}-p(y'\;|\;\v{x}_d,\Theta)\right\}\,p(h_i=1\;|\;\v{x}_d,y',\Theta)
\nonumber\\&=&
\sum_{y=1}^{C}\frac{\partial\c{L}_{\c{Y}|\c{X}}}{\partial u_{iy}}
\,,
\end{eqnarray}
and the gradient with respect to $\v{w}_i$ is
\begin{eqnarray}
\frac{\partial\c{L}_{\c{Y}|\c{X}}}{\partial \v{w}_{i}}
& = & 
\frac{1}{N}\sum_{d=1}^N\v{x}_d\sum_{y'=1}^{C}\left\{
\delta_{y',y_d}-p(y'\;|\;\v{x}_d,\Theta)\right\}\,p(h_i=1\;|\;\v{x}_d,y',\Theta)
\,.
\end{eqnarray}
Consequently, the discriminative log-likelihood $\c{L}_{\c{Y}|\c{X}}$ can be maximised using standard or accelerated
gradient ascent. Note, however, that the parameter $\v{a}$ from equation~\eqref{eq:energy}
does not appear in the RBC~\eqref{eq:p_y_x}, and therefore cannot be optimised discriminatively.

\end{document}
