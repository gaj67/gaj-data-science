\documentclass[a4paper]{article}
\usepackage{graphicx,subcaption}
\usepackage{amsmath,amsfonts}
\usepackage{qtree}
\title{Notes on Sequence Parsing}
\author{G.A. Jarrad}
\begin{document}
\maketitle
\numberwithin{equation}{section}
\numberwithin{figure}{section}
\numberwithin{table}{section}
\section{Introduction}\label{sec:intro}
The purpose of {\em sequence parsing} is to provide a hierarchically structured interpretation
of a sequence of tokens, 
$\vec{\tau}=(\tau_1,\tau_2,...,\tau_n)$.
Our primary example is the English sentence $\vec{\tau}=({\tt The},{\tt cat},{\tt sat},{\tt on},{\tt the},{\tt mat})$,
the parse of which has a variety of representations, as shown by (but not restricted to) 
Figures~\ref{fig:bracketing}--\ref{fig:parse-tree}. It is important to note, however, that some
natural languages are not so strictly ordered, and so a general sequence parse need not necessarily follow the same
ordering as the token sequence.
\begin{figure}[h]
\centering
\{
 \{{\tt The\, cat}\}\,
 \{{\tt sat}\,
  \{{\tt on}\,
   \{{\tt the\, mat}\}
  \}
 \}
\}
\caption{The parse represented as a hierarchical partitioning of the tokens.}
\label{fig:bracketing}
\end{figure}
\begin{figure}[h]
\centering
\(
\underset{\rho_5}{\underline{
   \underset{\rho_1}{\underline{\tt The\;\;cat}}\;\;
   \underset{\rho_4}{\underline{{\tt sat}\;\;
     \underset{\rho_3}{\underline{{\tt on}\;\;
       \underset{\rho_2}{\underline{\tt the\;\;mat}}
     }}
  }}
}}
\)
\caption{The parse represented as an ordered set of combination rules.}
\label{fig:nesting}
\end{figure}
\begin{figure}[h]
\centering
\Tree [.S [.NP [.DET The ] [.N cat ] ] [.VP [.V sat ] [.PP [.P on ] [.NP [.DET the ] [.N mat ] ] ] ] ]
\caption{The parse represented as a tree of nodes with  part-of-speech categories.}
\label{fig:parse-tree}
\end{figure}

Sequence parsing typically comprises two distinct stages: (i) {\em token analysis}, discussed more fully in 
Section~\ref{sec:token-analysis}; 
and (ii) {\em structure analysis}, discussed in Section~\ref{sec:structure-analysis}.
Briefly, the purpose of token analysis is to deduce, for each token $\tau_i$, the set $\Lambda(\tau_i)$ of {\em leaf nodes}
that represent plausible interpretations of the token. For example, in natural language understanding the leaf nodes
might include categories such as part-of-speech, as shown in Figure~\ref{fig:parse-tree}. 
The purpose of structure analysis is then to deduce a set $\Pi(\vec{\tau})$ of
rules that recursively combine sequences of nodes into higher-order {\em derived nodes}, 
until a single derived node spans the entire token sequence $\vec{\tau}$, 
again as shown in Figure~\ref{fig:parse-tree}.

In summary, $\Pi(\vec{\tau})$ may be characterised as a parse tree graph with a 
set ${\cal V}$ of nodes (both leaf and derived), and 
a set ${\cal R}$ of rules linking these nodes.
Each leaf node represents known information about the corresponding 
token, including the position of that token in the token sequence.
Likewise, each derived node represents a 
sequence of leaf and/or derived nodes, which includes knowledge of the positions of all corresponding tokens.
In general, therefore, every node $\nu\in{\cal V}$ {\em spans} a set of tokens. Specifically, we define
the span $\sigma(\nu)$ of node $\nu$ to be the ordered set of
indices of the underlying tokens. 
Consequently, each rule $\rho\in{\cal R}$ then takes the form
\begin{eqnarray}
\nu_{i_1}\;\nu_{i_2}\;\cdots\;\nu_{i_m} & \stackrel{\rho}{\rightarrow} & \nu_*\,,
\end{eqnarray}
where the derived node $\nu_*=\delta(\rho)$ combines the {\em predecessor} nodes 
$\vec{\pi}(\rho)=(\nu_{i_1},\ldots,\nu_{i_m})$
with a resulting ordered span of
\begin{eqnarray}
\sigma(\delta(\rho))=\bigcup_{\nu\in\vec{\pi}(\rho)}\sigma(\nu)\,.
\end{eqnarray}
 For example, if $\sigma(\nu_1)=\{2,1\}$ and $\sigma(\nu_2)=\{3\}$, then the rule
$\nu_1\;\nu_2\rightarrow\nu_3$ implies that $\sigma(\nu_3)=\{2,1,3\}$.

\end{document}
